\documentclass[a4paper]{article}
\usepackage{xeCJK}

\usepackage{geometry}
\geometry{left=3.2cm,right=3.2cm,top=3.2cm,bottom=3.2cm}

\setCJKmainfont{SimSun}
\linespread{1.3}

\usepackage{listings} 

\usepackage{indentfirst}
\setlength{\parindent}{2em} 

\newcommand{\titlefont}{\fontsize{14pt}{\baselineskip}\selectfont}
\newcommand{\docfont}{\fontsize{12pt}{\baselineskip}\selectfont}
\newcommand{\tablefont}{\fontsize{8pt}{\baselineskip}\selectfont}

\usepackage{float}
\usepackage{graphicx}

\usepackage{fancyhdr}
\pagestyle{fancy} \lhead{Laboratory 4 LC-2K Finite State Machine Simulator}\rhead{\thepage}
\author{13349047 计科一班 赖少凡}
\title{LC-2K Finite State Machine Simulator}

\begin{document}

\titlefont
\begin{center}
LC-2K Finite State Machine Simulator\\
\tablefont
\docfont
13349047~~~Computer Science~~~赖少凡\\
Email:Laishao\_yuan\@163.com 
Phone:18819461573(Short:61573)
\end{center}

\section{Basic design}
	Several blocks of the simulator:
	\begin{itemize}
		\item fetch block
			\begin{itemize}
				\item fetch: PC=PC+1 and load the instrReg
				\item fetch\_delay: wait for the clock
	    	 	  \end{itemize}
	    	\item branch: acts as the controler
	    	\item add
	    		\begin{itemize}
	    			\item add: load regA to ALU Operand
	    			\item add\_calc: load regB to bus and calculate the RegA+RegB
	    			\item add\_done: store the result into regDest
	    		\end{itemize}
	    	\item nand, nand\_calc, nand\_done: the same as add block
	    	\item lw
	    		\begin{itemize}
	    			\item lw: load the regA to ALU Operand
	    			\item lw\_addr: calculate the sum of regA and offset
	    			\item lw\_read: ready to read the Memory
	    			\item lw\_read\_delay: wait for the clock and store it into regB
	    		\end{itemize}
	    	\item sw, sw\_addr, sw\_write, sw\_write, sw\_write\_delay: the same as lw block
	    	\item beq
	    		\begin{itemize}
	    			\item beq: load the regA to ALU Operand
	    			\item beq\_calc: calculate the difference between regA and regB
	    			\item beq\_judge: jump according to (regA == regB)
	    			\item beq\_addr: calculate the address
	    			\item beq\_pc: set the PC
	    		\end{itemize}
	    	\item jalr
	    		\begin{itemize}
	    			\item jalr: regB = PC
	    			\item jalr\_a: PC = regA
	    		\end{itemize}
	    	\item halt: return
	    	\item noop: goto fetch
	\end{itemize}
\section{Optimize}
	To simplify the project, I use some macros:
	\begin{itemize}
		\item \begin{verbatim}
			define __STATE__(type) type: printState(&state, #type);
		\end{verbatim}
		\item \begin{verbatim}
		define _READMEM() memoryAccess(&state, 1)
		\end{verbatim}
		\item \begin{verbatim}
		define _WRITEMEM() memoryAccess(&state, 0)
		\end{verbatim}
		\item \begin{verbatim}#define _DISPATCH(opcode, label) if (((state.instrReg >> 22) 
		\end{verbatim}
		\begin{verbatim}
		 & 0x7) == opcode) goto label
		\end{verbatim}
		\item \begin{verbatim}
		define _REG_A state.reg[(state.instrReg >> 19) & 0x7]
		\end{verbatim}

		\item \begin{verbatim}
		#define _REG_B state.reg[(state.instrReg >> 16) & 0x7]
\end{verbatim}				
		\item \begin{verbatim}
		#define _REG_DEST state.reg[state.instrReg & 0x7]
\end{verbatim}						
		\item \begin{verbatim}
		#define _OFFSET convertNum(state.instrReg & 0x0000ffff)
\end{verbatim}				
	\end{itemize}\par
	Those optimization is all following the 10 rules given by the documentation.
	
\section{Result}
	The simulator is working as well as the simluator in the lab 02.
\end{document}